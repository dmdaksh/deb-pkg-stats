\documentclass[10pt,a4paper]{article}
\usepackage[margin=0.6in, top=0.7in, bottom=0.7in]{geometry} % Reduced margins for conciseness
\usepackage[T1]{fontenc}
\usepackage{mathptmx} % Times font for a classic look
\usepackage{hyperref}
\usepackage{enumitem}
\usepackage{booktabs} % For elegant tables
\usepackage{graphicx}
\usepackage{sectsty} % To customize section headings
\usepackage{calc} % For \widthof command
\usepackage{microtype} % Improves typography and spacing for elegance and conciseness

\sectionfont{\normalsize\bfseries} % Slightly smaller section titles
\subsectionfont{\small\bfseries}

\setlist{nosep,left=0pt, itemsep=1pt, topsep=2pt} % Tighter list spacing
\title{Package Statistics Command-Line Tool \\ \large Interview Test Submission Report \\ \normalsize \href{https://github.com/dmdaksh/deb-pkg-stats.git}{github.com/dmdaksh/deb-pkg-stats.git}}
\author{Daksh Maheshwari}
\date{\today}
\begin{document}
\pagestyle{empty}
\small % Use small font for the main body to fit more content
\maketitle
\vspace{-1.5em} % Reduce space after title

\section*{Overview}
\vspace{-0.5em} % Reduce space after section title
This report details the design, implementation, and testing of the \texttt{package\_statistics.py} command-line tool. It downloads Debian `Contents-<arch>.gz` indices, parses file-to-package mappings, aggregates file counts per package, and presents the top-N packages in a formatted table.

\section*{Key Features}
\vspace{-0.5em}
\begin{itemize}
	\item \textbf{Configurable:} Customize architecture, mirror, distribution, component, and top-N via CLI arguments.
	\item \textbf{Robust Parsing:} Handles large indices, paths with spaces, and skips malformed/header lines.
	\item \textbf{Error Handling:} Gracefully handles network errors and broken pipes (e.g., piping to `head`).
	\item \textbf{Clear Output:} Presents results in a formatted table with dynamic column widths.
	\item \textbf{Quality Assurance:} Type-annotated (Python 3.10+), formatted (\texttt{black}), linted (\texttt{ruff}), and type-checked (\texttt{mypy}).
\end{itemize}

\section*{Implementation Details}
\vspace{-0.5em}
Modular functions handle specific tasks:
\begin{description}[leftmargin=!,labelwidth=\widthof{\bfseries\texttt{download\_contents:}}, style=unboxed, itemsep=1pt]
	\item[\texttt{download\_contents:}] Fetches the index file using \texttt{urllib.request}, handling HTTP/URL errors.
	\item[\texttt{parse\_contents:}] Iterates decompressed lines, uses \texttt{str.rsplit(None, 1)} to separate path/packages, extracts base package names.
	\item[\texttt{get\_top\_packages:}] Orchestrates download, decompression (\texttt{gzip}), text wrapping (\texttt{TextIOWrapper}), parsing, and counting (\texttt{Counter}).
	\item[\texttt{main:}] Handles CLI arguments (\texttt{argparse}), logging, calls core logic, formats output table, and manages \texttt{BrokenPipeError}.
\end{description}

\section*{Testing \& Coverage}
\vspace{-0.5em}
Comprehensive \texttt{pytest} suite (\texttt{test\_package\_statistics.py}) covers parsing logic, download simulation (mocking), aggregation, and CLI behavior. Achieves \textbf{~93\% line coverage} (via \texttt{pytest-cov}).

\section*{Tools \& Configuration}
\vspace{-0.5em}
Project automation and configuration are centralized for consistency:
\begin{itemize}
	\item \textbf{Makefile:} Targets for install (\texttt{requirements-dev.txt}), format (\texttt{black}, \texttt{ruff --fix}), lint, typecheck, test, check, and clean.
	\item \textbf{pyproject.toml:} Unified configuration for \texttt{black}, \texttt{ruff} (lint), \texttt{isort}, \texttt{mypy}, and \texttt{pytest}.
	\item \textbf{requirements-dev.txt:} Defines dev dependencies: \texttt{black}, \texttt{ruff}, \texttt{mypy}, \texttt{pytest}, and \texttt{pytest-cov}.
	\item \textbf{.gitignore:} Excludes Python caches, coverage reports, virtual environments, OS files, and LaTeX auxiliary outputs.
\end{itemize}

\section*{Time Log}
\vspace{-0.5em}
\vspace{0.25em}
\centering
\begin{tabular}{@{}lr@{}}
	\toprule
	\textbf{Task}                     & \textbf{Time Spent} \\
	\midrule
	Requirements Review               & 0h 20m              \\
	Initial Coding                    & 0h 45m              \\
	Parsing Logic Refinement          & 0h 20m              \\
	CLI, Error Handling, Pipe Support & 0h 30m              \\
	Unit Tests Development            & 2h 15m              \\
	Test Coverage Improvements        & 0h 20m              \\
	Documentation (README, Report)    & 0h 30m              \\
	Makefile \& Build Enhancements    & 0h 15m              \\
	Output Formatting                 & 0h 25m              \\
	Final Debugging \& Cleanup        & 0h 20m              \\
	\midrule
	\textbf{Total Estimated Time}     & \textbf{6h}         \\
	\bottomrule
\end{tabular}

\end{document}
